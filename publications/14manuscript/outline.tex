
\documentclass{article}
\usepackage{outlines} 
\setlength{\topmargin}{-.5in} 
\setlength{\textheight}{9in}
\setlength{\oddsidemargin}{.125in} 
\setlength{\textwidth}{6in}

 \begin{document}

\title{The ``form of logic'' is conversation}
\author{MHT, NDG} \maketitle
 
\begin{outline}

\0 Manuscript outline
	
	\1 Introduction
		\2 Syllogistic reasoning is a testing grounds for theories of language and logic (or, reasoning)
		\2 Particular deviations from formal logic that are intriguing
			\3 Endorsement of conclusions for logically invalid arguments (Khemlani \& JL, 2012)
				\4 We derive a distribution over argument strengths from a probabilistic model
				\4 Compare with: Meta-analysis data / Rips data
			\3 Preference for some logically valid conclusions over others
				\4 We derive informativity effects from communicative principles (like PHM, but better)
				\4 Different from ``standard'' Grecian implicatures (Roberts, Newstead \& Griggs, 2001) because of the nature of the task
				\4 Compare with: Intuition (with simulations)
		\2 Background knowledge affects reasoning (belief bias)
			\3 Natural to consider with priors
				\4 Come up with schematic priors for Dube et al. materials
				\4 Experiments with different priors (Exp 1 \& 2?)
		\2 This approach is general to anything with semantics
			\3 Most \& few with PHM experiments
		\2 Discussion points
			\3 The role of language understanding / pragmatics in ``reasoning'' tasks
				\4 What does \emph{deduction} mean in the lexicon? (Umberto Eco has some words on this.)
			\3 The informational content of a syllogism \& optimal experiments
		\2 Outstanding questions / future directions
			\3 Can \emph{nothing} ever actually \emph{follow}?
				\4 Pragmatics of null utterances
			\3 Given-new effects in figural bias


\end{outline}

\end{document}