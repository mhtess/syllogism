\documentclass{llncs} %

\usepackage{pslatex}
\usepackage{apacite}
\usepackage{url}
\usepackage{graphicx}
\usepackage{listings}
\usepackage{color}
\usepackage{textcomp}
\usepackage{amsmath}
\usepackage{amssymb}
\usepackage{wrapfig}
\usepackage{lipsum}

\title{Questioning support for anomalous conclusions}


\author{{\large \bf Michael Henry Tessler} (mtessler@stanford.edu)}
\institute{Department of Psychology, Stanford University}
 
\begin{document}
\maketitle


\begin{abstract}
Here is the abstract
\end{abstract}

\section{Introduction}

\citeA{Tessler2014} proposed that syllogistic reasoning could be understood through the broader lens of language understanding. This intuition was formalized in a probabilistic model of pragmatic reasoning over syllogistic premises. To summarize the findings, they found the specifying the Question Under Discussion (QUD) as the quantifier relationship between the conclusion terms (i.e. the syllogistic conclusion) captured qualitative phenomena in syllogistic reasoning as well as providing a good overall fit to meta-analysis data. Here, I explore the implications of such a model, by considering the role of prior beliefs in syllogistic reasoning.

The syllogistic reasoning literature has highlighted an interaction between logic and belief \cite{Evans1983}. The proper characterization  of this interaction is open to debate \cite{Newstead1993, others}. Early on, \citeA{Evans1983} demonstrated that the \emph{a priori} believability of the conclusion influences the acceptability of the conclusion, and that this influence was more pronounced for invalid rather than valid syllogisms. However, when experimenters have included neutral materials for baseline comparisons, they find belief bias is primarily associated with \emph{rejecting unbelievable} conclusions particularly when the syllogism is invalid, leading some investigators to refer to it as ``belief debias'' \cite{Morley2004, Newstead1992}.

This interplay between logic and believability was followed up with a careful study by \citeA{Evans2001} which built upon an earlier finding by \citeA{Evans1999}. \citeA{Evans1999} conducted a standard syllogistic reasoning study (syllogistic reasoning with only abstract terms) and found a gradient of endorsement rates for logically invalid syllogisms. The authors dubbed these ``possible weak'' (PW) and ``possible strong'' (PS) syllogisms. \citeA{Evans2001} replicated this finding and found that PW syllogisms were more susceptible to positive belief bias (i.e. accepting a believable conclusion) and PS syllogisms were more susceptible to negative belief belief (i.e. rejecting an unbelievable conclusion). To summarize, the most fine-grained evidence is for a gradient acceptability of syllogisms, and a resulting gradient of influence of beliefs. 

\section{Probabilistic pragmatics in syllogistic reasoning}




\section{Optimal experiments for belief bias}

\subsection{OED with formal models}

\subsection{Overview of experiments} 

\section{Experiment 1: Background knowledge in syllogistic reasoning}

\subsection{Modeling results}

\section{Experiment 2: Pragmatics and background knowledge}

\subsection{Modeling results}

\section{Revisiting RSA: Wonky worlds}

\subsection{Re-analysis of Experiments 1 and 2}

\section{Discussion}

\section{Conclusion}

\end{document}
